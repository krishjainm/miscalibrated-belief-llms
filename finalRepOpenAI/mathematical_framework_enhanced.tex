\documentclass{article}
\usepackage{graphicx}
\usepackage{amsmath}
\usepackage{amsfonts}
\usepackage{amsthm}

\title{WMAC 2026: Emergent Communication Protocols in Multi-Agent Strategic Games}
\author{Harutyun Harry Ilanyan}
\date{October 2025}

\newtheorem{definition}{Definition}
\newtheorem{theorem}{Theorem}
\newtheorem{proposition}{Proposition}

\begin{document}

\maketitle

\section{Formal Framework}

We consider a partially observable multi-agent environment represented as a tuple 
\[
\mathcal{G} = (\mathcal{S}, \mathcal{A}_1, \ldots, \mathcal{A}_N, \mathcal{M}, T, R_1, \ldots, R_N, \Omega),
\]
where:
\begin{itemize}
    \item $\mathcal{S}$ is the global state space.
    \item Each agent $i \in \{1,\ldots,N\}$ observes a private state $s_i \in \mathcal{S}_i$ (e.g., private cards in poker).
    \item Each agent has an action space $\mathcal{A}_i$ and a message space $\mathcal{M}_i \subseteq \mathcal{M}$.
    \item $T$ is a stochastic transition kernel over states given actions.
    \item $R_i(s, a_1, \ldots, a_N)$ is the reward (or payoff) for agent $i$.
    \item $\Omega$ represents external constraints (e.g., banned phrases, lexical limitations).
\end{itemize}

At each timestep $t$, agent $i$ samples a message $m_i^t \sim \pi_i^m(\cdot \mid h_i^t, \Omega)$ and an action 
$a_i^t \sim \pi_i^a(\cdot \mid h_i^t, m_{-i}^t)$,
where $h_i^t$ denotes its local history, including observations, previous actions, and messages.  
The vector $m_{-i}^t$ represents the messages received from all other agents at time $t$.

\subsection{Communication Channel}
The communication mechanism defines a (possibly learned) mapping
\[
C: (s_1, \ldots, s_N, \Omega) \mapsto (m_1, \ldots, m_N)
\]
that induces a joint distribution $p(m, s, a \mid \Omega)$ over messages, states, and actions under the agents' policies $\pi = (\pi_1, \ldots, \pi_N)$.
Initially, $\mathcal{M}$ has no prescribed semantics: any structure in $p(m, s, a \mid \Omega)$ must emerge through optimization dynamics.

\section{Emergent Communication}

\begin{definition}[Emergent Communication Protocol]
A communication protocol $\pi^{comm} = \{\pi_i^m, \pi_i^a\}_{i=1}^N$ is said to exhibit 
\textbf{emergent communication} if it satisfies the following conditions for at least one pair of agents $(i,j)$:

\begin{enumerate}
    \item \textbf{Informational Dependence:}
    \[
    I(m_j; s_j) > \epsilon_1
    \quad \text{and/or} \quad
    I(m_j; s_{-j}) > \epsilon_1
    \]
    where $\epsilon_1 > 0$ is a significance threshold, and the emitted message carries mutual information about an agent's hidden or local state.

    \item \textbf{Behavioral Influence:}
    \[
    I(m_j; a_i \mid s_i) > \epsilon_2
    \]
    where $\epsilon_2 > 0$, i.e., another agent's action is conditionally dependent on the message, given its own observation.

    \item \textbf{Utility Improvement:}
    \[
    \mathbb{E}_{\pi^{comm}}[R_i] - \mathbb{E}_{\pi^{no\text{-}comm}}[R_i] > \epsilon_3
    \]
    where $\epsilon_3 > 0$, i.e., the introduction of the communication channel strictly improves expected utility relative to a no-communication baseline.

    \item \textbf{Protocol Stability:}
    \[
    \text{Var}_{\pi^{comm}}[I(m_j; s_j)] < \delta
    \]
    where $\delta > 0$ is a stability threshold, ensuring the protocol maintains consistent informational content over time.
\end{enumerate}

If all four criteria hold jointly, the system demonstrates \emph{non-trivial, beneficial, interpretable, and stable} emergent communication.
\end{definition}

\subsection{Protocol Adaptation Under Constraints}

\begin{definition}[Constraint-Resilient Emergence]
A protocol $\pi^{comm}$ exhibits \textbf{constraint-resilient emergence} if, when subjected to external constraints $\Omega' \subset \Omega$, it maintains emergent properties through adaptation:

\[
\exists \pi^{comm'} : \mathbb{E}_{\pi^{comm'}}[I(m_j; s_j \mid \Omega')] \geq (1-\alpha) \mathbb{E}_{\pi^{comm}}[I(m_j; s_j \mid \Omega)]
\]

where $\alpha \in [0,1]$ represents the maximum acceptable information loss under constraints.
\end{definition}

\subsection{Protocol Complexity and Sophistication}

\begin{definition}[Protocol Sophistication]
The \textbf{sophistication} of an emergent protocol is measured by:

\[
\Sigma(\pi^{comm}) = H(m \mid s) + \lambda \cdot \mathbb{E}[I(m; a \mid s)]
\]

where $H(m \mid s)$ captures the entropy of messages given states (protocol richness), and $\lambda$ weights the behavioral influence term.
\end{definition}

\section{Empirical Validation Framework}

\subsection{Statistical Tests for Emergence}

\begin{proposition}[Empirical Emergence Detection]
Given empirical data $\mathcal{D} = \{(m^t, s^t, a^t)\}_{t=1}^T$ from $K$ independent simulations, we can test for emergent communication using:

\begin{enumerate}
    \item \textbf{Mutual Information Test:}
    \[
    \hat{I}(m_j; s_j) = \frac{1}{K} \sum_{k=1}^K \hat{I}_k(m_j; s_j) \pm z_{\alpha/2} \cdot \text{SE}(\hat{I})
    \]
    where $\hat{I}_k$ is the empirical mutual information estimate for simulation $k$.

    \item \textbf{Conditional Independence Test:}
    \[
    \chi^2 = \sum_{m,a,s} \frac{(\hat{p}(m,a,s) - \hat{p}(m,s)\hat{p}(a|s))^2}{\hat{p}(m,s)\hat{p}(a|s)}
    \]
    Tests the null hypothesis $H_0: p(m,a|s) = p(m|s)p(a|s)$.

    \item \textbf{Utility Difference Test:}
    \[
    t = \frac{\bar{R}^{comm} - \bar{R}^{no\text{-}comm}}{\sqrt{\frac{s_1^2}{n_1} + \frac{s_2^2}{n_2}}}
    \]
    where $\bar{R}^{comm}$ and $\bar{R}^{no\text{-}comm}$ are sample means of rewards.
\end{enumerate}
\end{proposition}

\subsection{Protocol Evolution Metrics}

\begin{definition}[Protocol Convergence Rate]
The \textbf{convergence rate} of emergent protocols is measured by:

\[
\rho(t) = \frac{1}{K} \sum_{k=1}^K \frac{|\hat{I}_k^t(m; s) - \hat{I}_k^{t-1}(m; s)|}{\hat{I}_k^{t-1}(m; s)}
\]

Protocol convergence occurs when $\rho(t) < \epsilon$ for $t > T_{conv}$.
\end{definition}

\subsection{Adaptation Robustness Measures}

\begin{definition}[Adaptation Efficiency]
The \textbf{adaptation efficiency} under constraint $\Omega'$ is:

\[
\eta(\Omega') = \frac{\mathbb{E}[I(m'; s' \mid \Omega')]}{\mathbb{E}[I(m; s \mid \Omega)]} \cdot \frac{1}{T_{adapt}}
\]

where $T_{adapt}$ is the time to reach stable performance under constraints.
\end{definition}

\end{document}
